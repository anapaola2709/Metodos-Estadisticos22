% Options for packages loaded elsewhere
\PassOptionsToPackage{unicode}{hyperref}
\PassOptionsToPackage{hyphens}{url}
%
\documentclass[
]{article}
\title{examen-1.R}
\author{Usuario}
\date{2022-10-12}

\usepackage{amsmath,amssymb}
\usepackage{lmodern}
\usepackage{iftex}
\ifPDFTeX
  \usepackage[T1]{fontenc}
  \usepackage[utf8]{inputenc}
  \usepackage{textcomp} % provide euro and other symbols
\else % if luatex or xetex
  \usepackage{unicode-math}
  \defaultfontfeatures{Scale=MatchLowercase}
  \defaultfontfeatures[\rmfamily]{Ligatures=TeX,Scale=1}
\fi
% Use upquote if available, for straight quotes in verbatim environments
\IfFileExists{upquote.sty}{\usepackage{upquote}}{}
\IfFileExists{microtype.sty}{% use microtype if available
  \usepackage[]{microtype}
  \UseMicrotypeSet[protrusion]{basicmath} % disable protrusion for tt fonts
}{}
\makeatletter
\@ifundefined{KOMAClassName}{% if non-KOMA class
  \IfFileExists{parskip.sty}{%
    \usepackage{parskip}
  }{% else
    \setlength{\parindent}{0pt}
    \setlength{\parskip}{6pt plus 2pt minus 1pt}}
}{% if KOMA class
  \KOMAoptions{parskip=half}}
\makeatother
\usepackage{xcolor}
\IfFileExists{xurl.sty}{\usepackage{xurl}}{} % add URL line breaks if available
\IfFileExists{bookmark.sty}{\usepackage{bookmark}}{\usepackage{hyperref}}
\hypersetup{
  pdftitle={examen-1.R},
  pdfauthor={Usuario},
  hidelinks,
  pdfcreator={LaTeX via pandoc}}
\urlstyle{same} % disable monospaced font for URLs
\usepackage[margin=1in]{geometry}
\usepackage{color}
\usepackage{fancyvrb}
\newcommand{\VerbBar}{|}
\newcommand{\VERB}{\Verb[commandchars=\\\{\}]}
\DefineVerbatimEnvironment{Highlighting}{Verbatim}{commandchars=\\\{\}}
% Add ',fontsize=\small' for more characters per line
\usepackage{framed}
\definecolor{shadecolor}{RGB}{248,248,248}
\newenvironment{Shaded}{\begin{snugshade}}{\end{snugshade}}
\newcommand{\AlertTok}[1]{\textcolor[rgb]{0.94,0.16,0.16}{#1}}
\newcommand{\AnnotationTok}[1]{\textcolor[rgb]{0.56,0.35,0.01}{\textbf{\textit{#1}}}}
\newcommand{\AttributeTok}[1]{\textcolor[rgb]{0.77,0.63,0.00}{#1}}
\newcommand{\BaseNTok}[1]{\textcolor[rgb]{0.00,0.00,0.81}{#1}}
\newcommand{\BuiltInTok}[1]{#1}
\newcommand{\CharTok}[1]{\textcolor[rgb]{0.31,0.60,0.02}{#1}}
\newcommand{\CommentTok}[1]{\textcolor[rgb]{0.56,0.35,0.01}{\textit{#1}}}
\newcommand{\CommentVarTok}[1]{\textcolor[rgb]{0.56,0.35,0.01}{\textbf{\textit{#1}}}}
\newcommand{\ConstantTok}[1]{\textcolor[rgb]{0.00,0.00,0.00}{#1}}
\newcommand{\ControlFlowTok}[1]{\textcolor[rgb]{0.13,0.29,0.53}{\textbf{#1}}}
\newcommand{\DataTypeTok}[1]{\textcolor[rgb]{0.13,0.29,0.53}{#1}}
\newcommand{\DecValTok}[1]{\textcolor[rgb]{0.00,0.00,0.81}{#1}}
\newcommand{\DocumentationTok}[1]{\textcolor[rgb]{0.56,0.35,0.01}{\textbf{\textit{#1}}}}
\newcommand{\ErrorTok}[1]{\textcolor[rgb]{0.64,0.00,0.00}{\textbf{#1}}}
\newcommand{\ExtensionTok}[1]{#1}
\newcommand{\FloatTok}[1]{\textcolor[rgb]{0.00,0.00,0.81}{#1}}
\newcommand{\FunctionTok}[1]{\textcolor[rgb]{0.00,0.00,0.00}{#1}}
\newcommand{\ImportTok}[1]{#1}
\newcommand{\InformationTok}[1]{\textcolor[rgb]{0.56,0.35,0.01}{\textbf{\textit{#1}}}}
\newcommand{\KeywordTok}[1]{\textcolor[rgb]{0.13,0.29,0.53}{\textbf{#1}}}
\newcommand{\NormalTok}[1]{#1}
\newcommand{\OperatorTok}[1]{\textcolor[rgb]{0.81,0.36,0.00}{\textbf{#1}}}
\newcommand{\OtherTok}[1]{\textcolor[rgb]{0.56,0.35,0.01}{#1}}
\newcommand{\PreprocessorTok}[1]{\textcolor[rgb]{0.56,0.35,0.01}{\textit{#1}}}
\newcommand{\RegionMarkerTok}[1]{#1}
\newcommand{\SpecialCharTok}[1]{\textcolor[rgb]{0.00,0.00,0.00}{#1}}
\newcommand{\SpecialStringTok}[1]{\textcolor[rgb]{0.31,0.60,0.02}{#1}}
\newcommand{\StringTok}[1]{\textcolor[rgb]{0.31,0.60,0.02}{#1}}
\newcommand{\VariableTok}[1]{\textcolor[rgb]{0.00,0.00,0.00}{#1}}
\newcommand{\VerbatimStringTok}[1]{\textcolor[rgb]{0.31,0.60,0.02}{#1}}
\newcommand{\WarningTok}[1]{\textcolor[rgb]{0.56,0.35,0.01}{\textbf{\textit{#1}}}}
\usepackage{graphicx}
\makeatletter
\def\maxwidth{\ifdim\Gin@nat@width>\linewidth\linewidth\else\Gin@nat@width\fi}
\def\maxheight{\ifdim\Gin@nat@height>\textheight\textheight\else\Gin@nat@height\fi}
\makeatother
% Scale images if necessary, so that they will not overflow the page
% margins by default, and it is still possible to overwrite the defaults
% using explicit options in \includegraphics[width, height, ...]{}
\setkeys{Gin}{width=\maxwidth,height=\maxheight,keepaspectratio}
% Set default figure placement to htbp
\makeatletter
\def\fps@figure{htbp}
\makeatother
\setlength{\emergencystretch}{3em} % prevent overfull lines
\providecommand{\tightlist}{%
  \setlength{\itemsep}{0pt}\setlength{\parskip}{0pt}}
\setcounter{secnumdepth}{-\maxdimen} % remove section numbering
\ifLuaTeX
  \usepackage{selnolig}  % disable illegal ligatures
\fi

\begin{document}
\maketitle

\begin{Shaded}
\begin{Highlighting}[]
\CommentTok{\#APPT}
\CommentTok{\#12/10/22}
\CommentTok{\#Examen 1}

\FunctionTok{library}\NormalTok{(repmis)}
\end{Highlighting}
\end{Shaded}

\begin{verbatim}
## Warning: package 'repmis' was built under R version 4.1.3
\end{verbatim}

\begin{Shaded}
\begin{Highlighting}[]
\NormalTok{suelo }\OtherTok{\textless{}{-}} \FunctionTok{read.csv}\NormalTok{(}\StringTok{"CLASES/examendatosnuevos.csv"}\NormalTok{)}
\FunctionTok{View}\NormalTok{(suelo)}

\FunctionTok{boxplot}\NormalTok{(suelo}\SpecialCharTok{$}\NormalTok{nem }\SpecialCharTok{\textasciitilde{}}\NormalTok{ suelo}\SpecialCharTok{$}\NormalTok{suelo,}
        \AttributeTok{xlab =} \StringTok{"Tipos de suelo"}\NormalTok{,}
        \AttributeTok{ylab =} \StringTok{"Numero de nematodos"}\NormalTok{,}
        \AttributeTok{col =} \StringTok{"blue"}\NormalTok{)}
\end{Highlighting}
\end{Shaded}

\includegraphics{examen-1_files/figure-latex/unnamed-chunk-1-1.pdf}

\begin{Shaded}
\begin{Highlighting}[]
\FunctionTok{tapply}\NormalTok{(suelo}\SpecialCharTok{$}\NormalTok{nem, suelo}\SpecialCharTok{$}\NormalTok{suelo, var)}
\end{Highlighting}
\end{Shaded}

\begin{verbatim}
##    S1    S2    S3    S4    S5 
## 571.7 302.7 285.8 189.3  90.8
\end{verbatim}

\begin{Shaded}
\begin{Highlighting}[]
\FunctionTok{tapply}\NormalTok{(suelo}\SpecialCharTok{$}\NormalTok{nem, suelo}\SpecialCharTok{$}\NormalTok{suelo, mean)}
\end{Highlighting}
\end{Shaded}

\begin{verbatim}
##    S1    S2    S3    S4    S5 
## 148.8 140.8 130.4 100.4 161.6
\end{verbatim}

\begin{Shaded}
\begin{Highlighting}[]
\CommentTok{\#S1    S2    S3    S4    S5 }
\CommentTok{\#571.7 302.7 285.8 189.3  90.8}
\CommentTok{\#Cuántas veces es la diferencia entre la varianza más pequeña y la más grande?}
\FloatTok{571.7}\SpecialCharTok{/}\FloatTok{90.8}
\end{Highlighting}
\end{Shaded}

\begin{verbatim}
## [1] 6.296256
\end{verbatim}

\begin{Shaded}
\begin{Highlighting}[]
\CommentTok{\#6.296256}


\CommentTok{\#Cuáles serían las hipótesis nula y alternativa?}
\NormalTok{par.aov }\OtherTok{\textless{}{-}} \FunctionTok{aov}\NormalTok{(suelo}\SpecialCharTok{$}\NormalTok{nem }\SpecialCharTok{\textasciitilde{}}\NormalTok{ suelo}\SpecialCharTok{$}\NormalTok{suelo)}
\FunctionTok{summary}\NormalTok{(par.aov)}
\end{Highlighting}
\end{Shaded}

\begin{verbatim}
##             Df Sum Sq Mean Sq F value   Pr(>F)    
## suelo$suelo  4  10701  2675.2   9.287 0.000207 ***
## Residuals   20   5761   288.1                     
## ---
## Signif. codes:  0 '***' 0.001 '**' 0.01 '*' 0.05 '.' 0.1 ' ' 1
\end{verbatim}

\begin{Shaded}
\begin{Highlighting}[]
\CommentTok{\#Ho= No hay diferencias significativas}
\CommentTok{\#H1= Si hay diferencias significativas }

\CommentTok{\#Describe los resultados obtenidos indicando cuál es el valor del estadístico de contraste (F), los gradosde libertad del factor, los grados de libertad residuales y el valor de P.}
\CommentTok{\#Contraste F= 0.000207}
\CommentTok{\#Grados de libertad del factor= 4}
\CommentTok{\#Grados de libertad residuales= 20}
\CommentTok{\#Valor de p= 9.287}



\CommentTok{\#Si existen diferencias significativas, realiza una prueba de Tukey e indica donde existen diferenciassignificativas.}
\FunctionTok{TukeyHSD}\NormalTok{(par.aov)}
\end{Highlighting}
\end{Shaded}

\begin{verbatim}
##   Tukey multiple comparisons of means
##     95% family-wise confidence level
## 
## Fit: aov(formula = suelo$nem ~ suelo$suelo)
## 
## $`suelo$suelo`
##        diff         lwr        upr     p adj
## S2-S1  -8.0 -40.1208794  24.120879 0.9429980
## S3-S1 -18.4 -50.5208794  13.720879 0.4481002
## S4-S1 -48.4 -80.5208794 -16.279121 0.0017871
## S5-S1  12.8 -19.3208794  44.920879 0.7555248
## S3-S2 -10.4 -42.5208794  21.720879 0.8658492
## S4-S2 -40.4 -72.5208794  -8.279121 0.0095500
## S5-S2  20.8 -11.3208794  52.920879 0.3307073
## S4-S3 -30.0 -62.1208794   2.120879 0.0743745
## S5-S3  31.2  -0.9208794  63.320879 0.0595156
## S5-S4  61.2  29.0791206  93.320879 0.0001237
\end{verbatim}

\begin{Shaded}
\begin{Highlighting}[]
\FunctionTok{plot}\NormalTok{(}\FunctionTok{TukeyHSD}\NormalTok{(par.aov))}
\end{Highlighting}
\end{Shaded}

\includegraphics{examen-1_files/figure-latex/unnamed-chunk-1-2.pdf}

\begin{Shaded}
\begin{Highlighting}[]
\CommentTok{\#S2{-}S1  {-}8.0 {-}40.1208794  24.120879 0.9429980{-} no hay diferencias signficativas}
\CommentTok{\#S3{-}S1 {-}18.4 {-}50.5208794  13.720879 0.4481002{-} no hay diferencias significativas}
\CommentTok{\#S4{-}S1 {-}48.4 {-}80.5208794 {-}16.279121 0.0017871{-} si hay diferencias significativas }
\CommentTok{\#S5{-}S1  12.8 {-}19.3208794  44.920879 0.7555248{-} no hay diferencias significativas}
\CommentTok{\#S3{-}S2 {-}10.4 {-}42.5208794  21.720879 0.8658492{-} no hay diferencias significativas}
\CommentTok{\#S4{-}S2 {-}40.4 {-}72.5208794  {-}8.279121 0.0095500{-} si hay diferencias significativas}
\CommentTok{\#S5{-}S2  20.8 {-}11.3208794  52.920879 0.3307073{-} no hay diferencias significativas}
\CommentTok{\#S4{-}S3 {-}30.0 {-}62.1208794   2.120879 0.0743745{-} no hay diferencias significativas}
\CommentTok{\#S5{-}S3  31.2  {-}0.9208794  63.320879 0.0595156{-} no hay diferencias significativas}
\CommentTok{\#S5{-}S4  61.2  29.0791206  93.320879 0.0001237{-} si hay diferencias significativas }



\CommentTok{\#Conclusion}
\CommentTok{\#Algunas de las variables tienen una diferencia significativa. Sin embargo son muy pocas, la mayoria no tiene una diferencia muy grande }
\end{Highlighting}
\end{Shaded}


\end{document}
